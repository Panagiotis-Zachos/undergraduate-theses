%!TEX root = ../main.tex

\chapter{Συμπεράσματα}

Οι μέθοδοι για την εκτίμηση γωνίας άφιξης βασίζονται όλο και περισσότερο σε τεχνικές που χρησιμοποιούν μηχανική μάθηση, κυρίως νευρωνικά δίκτυα, που μέχρι στιγμής χρησιμοποιούν διαφορετικές προσεγγίσεις για τα δεδομένα εκπαίδευσης, τις αρχιτεκτονικές και μετρικές απόδοσης. Τέτοιες μέθοδοι συνήθως χρησιμοποιούν δεδομένα από φασματογραφήματα πλάτους ή/και φάσης.

Μια βασική απαίτηση όλων αυτών των μεθόδων είναι η αποτελεσματική χρήση των εξαχθέντων παραμέτρων από κατάλληλα σήματα, για εύρωστη και χαμηλής πολυπλοκότητας εκπαίδευση των νευρωνικών δικτύων. Τυπικά τέτοιες παράμετροι είναι οι αμφιωτικές παράμετροι ILD και ITD.

Σε αυτή την εργασία χρησιμοποιείται ένα fully connected και ένα convolutional νευρωνικό που χρησιμοποιούν μια νέα προσέγγιση για τη συμπίεση των εξαχθέντων παραμέτρων, που επιτρέπει την αποτελεσματική αναπαράστασή των ILD και ITD, για τη γρήγορη και αξιόπιστη εκπαίδευσή των μοντέλων λαμβάνοντας υπόψιν ταυτόχρονα το γεγονός ότι οι υπολογιστικοί πόροι και τα dataset είναι συνήθως περιορισμένα. Κατά την λειτουργία σε πραγματικές συνθήκες, όπως παραμέτροοι οι οποίες εξάγονται από ακουστικά σήματα που παράγονται μέσα σε δωμάτια με αντήχηση, τότε το μέγεθος των παραμέτρων καθίσταται απαγορευτικό. Με κίνητρο αυτό το γεγονός, η εργασία αυτή εισάγει μια καινούρια μέθοδο προεπεξεργασίας των αμφιωτικών παραμέτρων, η οποία απλοποιεί τα αρχικά πολυδιάστατα δεδομένα σε μόνο δύο διαστάσεις, που αποτελούν μια συνοπτική, οπτική περιγραφή τους, τα προφίλ.

Κατά τη φάση του testing, χρησιμοποιήθηκαν δεδομένα που προέρχονται από πραγματικά δωμάτια με αντήχηση και τα μοντέλα επιτυγχάνουν αποτελέσματα με υψηλή ακρίβεια, από σήματα εισόδου που είναι μόλις $200msec$. Η ακρίβεια παραμένει υψηλή ακόμα και όταν τα σήματα είναι διαφορετικά από αυτά που χρησιμοποιήθηκαν κατά την εκπαίδευση των μοντέλων.

Οι μέχρι στιγμής δοκιμές έχουν δώσει αισιόδοξα αποτελέσματα, επιβεβαιώνοντας πως η προτεινόμενη μέθοδος προσέγγισης του προβλήματος της εκτίμησης DOA, λειτουργεί ακόμα και στην περίπτωση πολύ χαμηλότερης δειγματοληψίας, όπου τα δεδομένα είναι ακόμα πιο περιορισμένα. Τα αποτελέσματα σε αυτή την περίπτωση ορισμένες φορές ξεπερνούν αυτά που έχουν παρουσιαστεί σε αυτή την εργασία.

Ο έλεγχος της απόδοσης της μεθόδου στην περίπτωση που τα δεδομένα προέρχονται από ένα δωμάτιο που δεν έχει χρησιμοποιηθεί κατά την εκπαίδευση αποτελεί αντικείμενο μελλοντικής μελέτης, όπως επίσης και η τροφοδότηση του μοντέλου με περισσότερα δωμάτια, καθώς και διαφορετικές θέσης του ακροατή μέσα σε αυτά για την επίτευξη πιο εύρωστης λειτουργίας.