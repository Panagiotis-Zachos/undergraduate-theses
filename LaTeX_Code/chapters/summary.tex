\pagestyle{plain}
\begin{center}
{\LARGE Περίληψη}\\[1cm]
\end{center}

Στην παρούσα Διπλωματική Εργασία προτείνεται μια νέα μέθοδος συμπίεσης των διαστάσεων των αμφιωτικών παραμέτρων ILD και ITD. Ο στόχος της μεθόδου είναι η βέλτιστη αξιοποίηση αυτών των παραμέτρων για την εκπαίδευση Νευρωνικών Δικτύων καθώς και σε άλλες εφαρμογές μηχανικής μάθησης. Υλοποιούνται επίσης δύο μοντέλα Νευρωνικών Δικτύων, ένα πλήρως διασυνδεδεμένο με δομή που παρομοιάζει έναν πολυεπίπεδο Perceptron, καθώς και ένα CNN, με στόχο την εκτίμηση της γωνίας άφιξης μιας ακουστικής διέγερσης, σε πραγματικά δωμάτια με αντήχηση, από τις συμπιεσμένες παραμέτρους. Τα μοντέλα επιτυγχάνουν εξαιρετική ακρίβεια στην εκτίμηση με το μέσο λάθος να κυμαίνεται κάτω από $5^o$, αξιοποιώντας τις παραμέτρους που έχουν συμπιεστεί κατά έναν παράγοντα $\sim88\%$. Η προσέγγιση εκτίμησης της γωνίας άφιξης σε αυτή την εργασία, ξεπερνά παραδοσιακές μεθόδους εντοπισμού που αξιοποιούν τεχνικές μηχανικής μάθησης, τόσο στον χρόνο που χρειάζεται η εκπαίδευση του μοντέλου, όσο και στα αποτελέσματα που επιτυγχάνονται.

\begin{center}
    \rule{\textwidth}{1pt}
\end{center}{}
\textbf{Λέξεις-κλειδιά:} Αμφιωτικές παράμετροι, Νευρωνικά Δίκτυα, Εκτίμηση 
\\*DOA, Συμπίεση Δεδομένων.
\begin{center}
    \rule{\textwidth}{1pt}
\end{center}{}