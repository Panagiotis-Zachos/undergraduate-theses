%!TEX root = ../main.tex

\chapter{Εισαγωγή}
\markboth{Εισαγωγη}{}
%\vspace{-1.3in}


\noindent Τα προβλήματα εντοπισμού γωνίας άφιξης (Direction of Arrival - DOA) βασισμένα σε παρατηρήσεις που προκύπτουν από το σήμα που έχει εκπεμφθεί, είναι μια ιδιαίτερα σημαντική λειτουργία της χωρικής ακοής των ανθρώπων. Συνήθως αντιμετωπίζονται με μικροφωνικές διατάξεις, αλλά σε πολλές περιπτώσεις, μόνο δύο μικρόφωνα είναι διαθέσιμα (λ.χ. τοποθετημένα σε ένα τεχνητό ή πραγματικό κεφάλι), το οποίο καθιστά την εκτίμηση της DOA σημαντικά δυσκολότερο πρόβλημα, ιδιαίτερα σε περιβάλλοντα με έντονη αντήχηση, παρόλο που ένας άνθρωπος προσαρμόζεται γρήγορα σε τέτοιες αντίξοες συνθήκες, εντοπίζοντας τους ήχους με έναν εύρωστο τρόπο. Η εκτίμηση της θέσης μιας ακουστικής πηγής, μέσω ενός εκπεμπόμενου ηχητικού σήματος, έχει ευρύ φάσμα εφαρμογών, όπως επόμενης γενιάς εμφυτεύματα κοχλία \cite{Nicoletti2013, Aronoff2010}, ακουστικά βοηθήματα \cite{Grimaldi2019}, ρομπότ \cite{Tahmid2010}, συστήματα κατ' οίκον φροντίδας \cite{Chen2013} και αναγνώρισης ομιλίας \cite{Stern2012}. Ήδη υπάρχουσες στρατηγικές εντοπισμού πηγών, μπορούν να κατηγοριοποιηθούν με έναν ελαστικό τρόπο σε τρεις γενικές κατηγορίες: αυτές που βασίζονται στη μεγιστοποίηση της οδηγούμενης απόκρισης ισχύος (Steered Response Power - SRP) ενός beamformer \cite{Dibiase2001}, τεχνικές που υιοθετούν έννοιες υψηλής φασματικής ανάλυσης \cite{Johnson1993, Haykin1991} και προσεγγίσεις που αξιοποιούν την πληροφορία που προέρχεται από τις διαφορές χρόνων άφιξης (Time Difference of Arrival - TDOA) σε ζεύγη μικροφώνων \cite{Chen2006}. 

Οι περισσότερες μέθοδοι εκτίμησης DOA βασίζονται σε προσεγγίσεις που εντάσονται στην $ 3^\text{η} $ κατηγορία, δηλαδή στην αξιοποίηση του TDOA, με την μέθοδο Generalized Cross-Correlation PHAse Transform (GCC-PHAT) να είναι η επικρατέστερη \cite{Knapp1976}. Μια σύνοψη των τεχνικών TDOA βρίσκεται στο \cite{Chen2006}. Στην πράξη, όλες αυτές οι προσεγγίσεις μπορεί να περιορίζονται από ένα, ή συνδυασμό κάποιων μειονεκτημάτων: υψηλό υπολογιστικό κόστος, μη ρεαλιστικές παραδοχές για τα σήματα ή/και τον θόρυβο, αναξιόπιστη απόδοση σε πραγματικά περιβάλλοντα, ειδικά σε δωμάτια με έντονη αντήχηση. Η αυξημένη διαθεσιμότητα υπολογιστικής δύναμης, έχει επιτρέψει την εμφάνιση νέων μεθόδων, που χρησιμοποιούν αλγορίθμους μηχανικής μάθησης (Machine Learning - ML) που μπορούν να αντιμετωπίσουν το πρόβλημα της εκτίμησης DOA.

Τα συστήματα μηχανικής μάθησης που σχεδιάζονται για αυτό το σκοπό, χρησιμοποιούν διαφορετικά δεδομένα εκπαίδευσης, αρχιτεκτονικές, περιβάλλοντα δοκιμής και μετρικές αξιολόγησης. Η πιο ευρέως διαδεδομένη πλέον μέθοδος, είναι η επιβλεπόμενη μάθηση ή supervised machine learning, η οποία έχει ως στόχο, την εκμάθηση μιας συνάρτησης, ενώ ταυτόχρονα την βελτιστοποιεί, η οποία αντιστοιχίζει μια είσοδο σε μια έξοδο, βασιζόμενη σε παραδείγματα εισόδου-εξόδου που χρησιμοποιούνται κατά την εκπαίδευση. Η συνάρτηση εκτιμάται από labeled δεδομένα εκπαίδευσης, που αποτελούνται από ένα σύνολο παραδειγμάτων εκπαίδευσης \cite{Russel2010, Abadi2015}. Πολλές έρευνες στις οποίες εφαρμόζονται τέτοιες τεχνικές ταξινόμησης DOA και αλγόριθμοι εντοπισμού, έχουν δείξει πως αυτή η προσέγγιση είναι έγκυρη και παρέχει εξαιρετικά αποτελέσματα, πχ χρησιμοποιώντας την στρατηγική random forest ensemble \cite{Kamaris2016, Kamaris2017, Kamaris2018} καθώς και τεχνικές που αξιοποιούν νευρωνικά δίκτυα \cite{Zhang2019, Adavanne2017, Perotin2019}.

Σε αυτή την εργασία γίνεται προσπάθεια να αναπτυχθεί μια νέα προσέγγιση για τη μείωση δεδομένων που προέρχονται από αμφιωτικές παραμέτρους, με κατάλληλο τρόπο για εκτίμηση DOA με regression νευρωνικά δίκτυα. Οι εν λόγω αμφιωτικές παράμετροι, είναι η αμφιωτική διαφορά χρόνου άφιξης και η αμφιωτική διαφορά στάθμης ακουστικής πίεσης (Interaural Time Difference και Interaural Level Difference ITD και ILD αντίστοιχα). Συγκεκριμένα, οι αμφιωτικές παράμετροι προέρχονται από μετρήσεις Binaural Room Impulse Responses (BRIRs) από πραγματικά δωμάτια, όπου τα αντηχητικά αποτελέσματα μεγαλώνουν σημαντικά το μέγεθος του dataset ενώ ταυτόχρονα δεν υπάρχουν αρκετές βάσεις δεδομένων που να πληρούν τις απαιτήσεις που υπήρχαν ως προς την αξιοπιστία, το πλήθος των δεδομένων και την ομοιογένειά τους. Για τον σκοπό αυτό, εκτελέστηκε μια εκτενής συγκριτική μελέτη, κατά την οποία αξιολογήθηκε η απόδοση των νευρωνικών όταν χρησιμοποιήθηκαν οι προτεινόμενες συμπιεσμένες αμφιωτικές παράμετροι και οι ασυμπίεστες για τον ίδιο σκοπό, την εκτίμηση της γωνίας άφιξης. Δοκιμάστηκαν δύο εναλλακτικές αρχιτεκτονικές νευρωνικών: μία πλήρως διασυνδεδεμένη (FC), ουσιαστικά ένας Multilayer Perceptron, και μία συνελικτική (CNN).

Εδώ, οι μέθοδοι που δοκιμάστηκαν, επιχειρούν να προβλέψουν την συνάρτηση από τις μεταβλητές εισόδου, σε συνεχείς πραγματικές μεταβλητές δηλαδή την γωνία άφιξης, από τις αμφιωτικές παραμέτρους που χρησιμοποιούνται ως παραδείγματα εκπαίδευσης. Οι μετρικές που χρησιμοποιήθηκαν για να αξιολογηθεί η απόδοση του εκάστοτε μοντέλου είναι το Μέσο Τετραγωνικό Σφάλμα (Mean Squared Error - MSE), το οποίο χρησιμοποιήθηκε ως εκτιμήτρια συνάρτηση (Objective Function) κατά την εκπαίδευση,  η ρίζα του Μέσου Τετραγωνικού Σφάλματος (Root Mean Squared Error - RMSE) και το Μέσο Απόλυτο Σφάλμα (Mean Absolute Error - MAE) που παρέχουν μια εκτίμηση του πόσο 'μακριά' είναι η προβλεπόμενη DOA, από την πραγματική. 

Όπως έχει αναφερθεί, η μέθοδος συμπίεσης δεδομένων που προτείνεται εδώ, μειώνει δραματικά το πλήθος των διαστάσεων των αμφιωτικών παραμέτρων, επιτυγχάνοντας έναν εντυπωσιακό λόγο συμπίεσης έως και 97\% καθώς και εξαιρετικά αποτελέσματα εκπαίδευσης των Νευρωνικών Δικτύων. Κατ' επέκταση αυτό σημαίνει πως δεν χρειάζεται να γίνει κανένας συμβιβασμός μεταξύ της ταχύτητας σύγκλισης του μοντέλου και της ακρίβειάς του.

Το μεγαλύτερο μέρος της επεξεργασίας των δεδομένων έγινε σε περιβάλλον MATLAB, χρησιμοποιώντας την εργαλειοθήκη Auditory Modeling Toolbox \cite{Soendergaard2013}, ενώ η προεπεξεργασία προτού τα δεδομένα χρησιμοποιηθούν για την εκπαίδευση του Νευρωνικού επεξεργάστηκαν με τη γλώσσα Python, όπου έγινε και η εκπαίδευση χρησιμοποιώντας τη διεπαφή TensorFlow \cite{tensorflow2015-whitepaper}.